% ********** Приклад оформлення пояснювальної записки **********
% *********  до атестаційної роботи ступеня бакалавра **********
% ***  Виправлення Таврова Д.Ю. на основі файлу Яковлєва С.В. **

\documentclass{bachelor_thesis}

% Додаткові пакети вносьте у цей файл
%%%% У даний файл додавайте всі необхідні вам додаткові пакети, наприклад...


%%%% Діаграми
%\usepackage{tikz}                      % !!! невідомий конфлікт з якимось іншим пакетом

% Пакети для кольорових текстів (необхідні для команди \todo)
%\usepackage{xcolor}                     % !!! невідомий конфлікт з якимось іншим пакетом
%\usepackage{colortbl}

\usepackage{euscript}   % для красивих математичиних шрифтів \EuScript

%%%% ...і таке інше

%\usepackage[normalem]{ulem} % для підкреслень uline
%\ULdepth = 0.16em % визначаємо відстань від лінії до тексту вище/нижче

\usepackage{xstring} % For string manipulation


% Додаткові визначення та перевизначення команд вносьте у цей файл
\input{02_redefinitions}

% Відомості про автора роботи
%%% Основні відомості %%%
%\newcommand{\UDC}                      % УДК
%{(впишіть правильний УДК!)}            % УДК виглядає приблизно як 004.056.5 або 513.2, або навіть 004.056.5:513.2+519.1
% Для того, щоб знайти правильний УДК, використовуйте каталог https://teacode.com/online/udc/

\newcommand{\reportAuthor}             % ПІБ автора повністю
{Іванов Петро Сидорович}
\newcommand{\reportAuthorGen}          % ПІБ автора в родовому відмінку
{Іванову Петру Сидоровичу}
\newcommand{\reportAuthorShort}        % ПІБ автора коротко
{Петро ІВАНОВ}
\newcommand{\reportAuthorGroup}        % група автора
{КМ-N3}
\newcommand{\reportTitle}              % Назва роботи
{Назва роботи}
%% використовуйте символ "\par" або "\\" для розбиття назви на декілька рядків

\newcommand{\supervisorFio}            % Науковий керівник, ПІБ повністю
{Прізвище Ім'я По-батькові}
\newcommand{\supervisorFioShort}       % Науковий керівник, ПІБ коротко
{Ім'я ПРІЗВИЩЕ}
\newcommand{\supervisorRegalia}        % Науковий керівник: посада, степінь, звання
{посада, ступінь, звання}              % наприклад: доцент кафедри ПМА, канд. техн. наук, доцент

\newcommand{\reportOrder}{\quotes{23} квітня 2024 р. \No ????-С} % реквізити наказу

\newcommand{\applicationDate}          % термін подання роботи
{\quotes{10} червня 2024 р.}

\newcommand{\assignmentDate}           % дата видачі завдання
{\quotes{05} лютого 2024 р.}

\newcommand{\consultFio}               % Консультант, ПІБ повністю
{}
\newcommand{\consultRegalia}           % Консультант: звання, ступінь, посада
{}
% Якщо у вас нема консультанта - залишайте ці поля порожніми


\newcommand{\reviewerFio}              % Рецензент, ПІБ повністю
{Прізвище Ім'я По-батькові}                        
\newcommand{\reviewerRegalia}          % Рецензент: звання, ступінь, посада
{посада, ступінь, звання}

\newcommand{\YearOfDefence}            % рік захисту
{2024}
\newcommand{\YearOfBeginning}          
{2023}

% Це файл з інформацією про джерела, на які посилаються в роботі
% Якщо формувати бібліографію вручну, то це варто закоментувати
\addbibresource{thesis.bib}

% Починаємо верстку документа
\begin{document}

% Створюємо титульну сторінку
%!TEX root = main.tex

\thispagestyle{empty}
\setlength{\parindent}{0cm}

\begin{titlepage}
    \centering
    \vspace{2cm} 
    
    НАЦІОНАЛЬНИЙ ТЕХНІЧНИЙ УНІВЕРСИТЕТ УКРАЇНИ \\
    «КИЇВСЬКИЙ ПОЛІТЕХНІЧНИЙ ІНСТИТУТ \\
    ІМЕНІ ІГОРЯ СІКОРСЬКОГО»
        
    \begin{center}
        Факультет прикладної математики \\
        Кафедра прикладної математики
    \end{center}
    
    \vspace{2cm}
    
    \begin{center}
        Звіт \\
        із лабораторної роботи \\
        із дисципліни «\reportCourse» \\
        на тему: \\
        \reportTitle
    \end{center}
    
    \vspace{1.5cm}
    
    \begin{flushleft}
        \authorMake: \hfill \supervisorReviewed: \\
        \authorPerformed групи \reportAuthorGroup \hfill \supervisorRegalia \\
        \reportAuthor \hfill \supervisorFio
    \end{flushleft}
        
    \vfill
    \begin{center}
    {Київ~---~\YearOfDefence\ року}
    \end{center}
\end{titlepage}

\clearpage % починаємо з нової сторінки

\setcounter{page}{2} % номер сторінки буде 2

\setlength{\parindent}{1.25cm} % задаємо відступ першого рядка абзацу (5 знаків, або 1,25 см)

% пропуск рядка до та після формули
\setlength{\belowdisplayskip}{14pt}
\setlength{\abovedisplayskip}{14pt}
\setlength{\belowdisplayshortskip}{14pt}
\setlength{\abovedisplayshortskip}{14pt}


% Створюємо завдання
\pagestyle{empty}
\setlength{\parindent}{0cm}

{\centering
	\textbf{Національний технічний університет України}
	
	\textbf{\quotes{Київський політехнічний інститут імені Ігоря Сікорського}}

	\textbf{Факультет прикладної математики}
	
	\textbf{Кафедра прикладної математики}

}

{\raggedright
Рівень вищої освіти --- перший (бакалаврський)

Спеціальність --- 113~Прикладна математика

Освітньо-професійна програма \quotes{Наука про дані та математичне моделювання}

}

\addvspace{12pt} % невеликий відступ

\begin{flushright}
	\renewcommand{\arraystretch}{0.8}
	\begin{tabular}{l}
		\MakeUppercase{Затверджую} \\
		Завідувач кафедри\\
		\_\_\_\_\_\_\_\_\_\_\_\_\_~Олег ЧЕРТОВ \\
		\quotes{\_\_\_\_}~\_\_\_\_\_\_\_\_\_\_\_\_\_~\the\year~р. \\
	\end{tabular}
\end{flushright}

{\centering
	\textbf{\MakeUppercase{Завдання}}
	
	\textbf{на дипломну роботу студенту}
	
	%\addvspace{14pt}
	
	\reportAuthorGen

}

1. Тема роботи: \quotes{\reportTitle},
керівник роботи \supervisorFio, \supervisorRegalia,  затверджені наказом по університету \reportOrder.

2. Термін подання студентом роботи: \applicationDate

3. Вихідні дані до роботи: ???

4. Зміст роботи: ???

5. Перелік ілюстративного матеріалу: ???

%% Якщо немає консультантів, видаліть пункт 6. і зробіть пункт 7. пунктом 6.
6. Консультанти розділів роботи:

\addvspace{12pt}
\begin{center}
	\begin{tabularx}{\textwidth}{|C{0.2\textwidth}|L{0.4\textwidth}|C{0.15\textwidth}|C{0.15\textwidth}|}
		\hline
		Розділ & \multicolumn{1}{|C{0.4\textwidth}|}{Прізвище, ініціали та посада консультанта} & \multicolumn{2}{C{0.3\textwidth}|}{Підпис, дата} \\ \cline{3-4}
		&  & завдання видав & завдання прийняв \\
		\hline
		Математичне забезпечення & \consultFio, \consultRegalia & & \\
		\hline
	\end{tabularx}
\end{center}

\addvspace{12pt}

7. Дата видачі завдання: \assignmentDate

% Якщо перша частина завдання вилізе за сторінку --- приберіть команду \newpage
% Календарний план є продовженням завдання, а не окремою частиною
\newpage

\begin{center}
Календарний план

\addvspace{12pt}

    \begin{tabularx}{\textwidth}{|C{0.1\textwidth}|L{0.4\textwidth}|C{0.25\textwidth}|C{0.15\textwidth}|}
    \hline
    \No\ з/п & \multicolumn{1}{C{0.4\textwidth}|}{Назва етапів виконання дипломної роботи} & Термін виконання етапів роботи & Примітка \\
    \hline 
    % номер етапу
    1 & 
    % назва етапу
    Узгодження теми роботи із науковим керівником & 
    % термін виконання
    01-15 вересня \YearOfBeginning~р. &
    \\
%%% -- початок інтервалу для копіювання
    \hline 
    % номер етапу
    2 & 
    % назва етапу
    Огляд опублікованих джерел за тематикою дослідження & 
    % термін виконання
    Вересень-жовтень \YearOfBeginning~р. &
    \\
%%% -- кінець інтервалу для копіювання
% не прибирайте амперсанди та \\ наприкінці рядків!
% скопійовані інтервали вставляти перед фінальною \hline та заповнювати відповідно
% ось так:
%%% -- початок інтервалу для копіювання
    \hline 
    % номер етапу
    3 & 
    % назва етапу
    \ldots & 
    % термін виконання
    \ldots &
    \\
%%% -- кінець інтервалу для копіювання
    \hline %фінальна hline
    \end{tabularx}
\end{center}


\begin{tabular*}{\textwidth}{@{\extracolsep{\fill}}lrr}
	Студент &  \_\_\_\_\_\_\_\_\_\_\_\_\_ & \reportAuthorShort \\
	
	Керівник роботи & \_\_\_\_\_\_\_\_\_\_\_\_\_ & \supervisorFioShort \\
\end{tabular*}

\clearpage

\pagestyle{plain}

\setlength{\parindent}{1.25cm}


% У даному костильному рішенні перші три сторінки (титул та завдання на 
% роботу) друкуються окремо від основної частини тез.
% Тому перша сторінка сформованого документу нумерується як четверта

% Створюємо анотації
%!TEX root = ../abstract.tex

\abstractUkr

Дипломну роботу виконано на ?? аркушах, вона містить ?? додатки та перелік посилань на використані джерела з ?? найменувань. У роботі наведено ?? рисунків та ?? таблиць.

В анотації треба коротко сказати, що було зроблено в роботі. Як правило, це мета роботи та основні завдання, які було виконано для її досягнення. 

% наприкінці анотації потрібно зазначити ключові слова

Ключові слова: ???.

\abstractEng

% Тут треба навести переклад україномовної анотації

% Не прибирайте даний рядок

\clearpage

% Створюємо зміст
%\pagenumbering{gobble}
\tableofcontents
\cleardoublepage
%\pagenumbering{arabic}
%\setcounter{page}{8}    %!!! -- продумати, як автоматизувати номер сторінки

% Створюємо перелік умовних позначень, скорочень і термінів
\shortings
%!TEX root = ../thesis.tex
% створюємо перелік умовних позначень, скорочень і термінів у форматі 
% поняття --- розшифровка
ROI – return on investment.

GDPR – General Data Protection Regulation.

OR – open rate.




% Створюємо вступ
\intro
%!TEX root = ../main.tex
% створюємо вступ
Актуальність даної лабораторної роботи
у тому, що без її виконання вам буде важко закрити семестр з даного предмету((

Ну а враховуючи, що ви вирішили використовувати ЛаТеХ і навіть відкрити цей док -- скажу, що без написання стипендії скоріше за все не буде,
тому ноги в руки і за лабу )))


% Додаємо глави
%!TEX root = ./main.tex

\chapter{Постановка задачі}
\label{chap:problem}  %% відмічайте кожен розділ певною міткою -- на неї наприкінці необхідно посилатись

Першим розділом лабораторної роботи зазвичай є розділ про постановку задачі, але якщо то буде не він - бити ніхто не буде.

Тут треба вказати, серед іншого, мету і завдання.

\textit{Метою дослідження} є певна абстрактна недосяжна річ на кшталт 
загальнолюдського щастя на горизонті. Для досягнення мети потрібно 
розв'язати \textit{задачу дослідження}, яка полягає у чомусь суттєво більш 
конкретному.


%!TEX root = ../thesis.tex

\chapter{(Назва другого розділу)}
\label{chap:review}

\section{(Назва першого підрозділу)}

Другий розділ, як правило, повинен бути присвячений огляду попередніх результатів за 
тематикою вашого дослідження. У даному розділі повинні міститись всі 
визначення та описи, потрібні для дальшого викладення матеріалу, та результати 
ваших попередників.

Абсолютно на всі не ваші результати повинні стояти 
в належний спосіб оформлені посилання.

Розмір другого (оглядового) розділу не повинен перевищувати третини вашої 
дипломної роботи (без урахування додатків).


\section{(Назва другого підрозділу)}

Наведемо основні правила оформлення текстів у системі \LaTeX.

Для абзацу робіть пусті рядки у файлі. Курсивний текст робиться командою 
\texttt{textit}: \textit{ось так}. 

\quotes{Лапки} робляться командою \texttt{quotes}. Довге 
тире у тексті --- трьома дефісами, коротке -- двома дефісами; у формулах 
мінуси робляться одним дефісом: $a-b$.

Пишіть звичайний текст звичайним текстом, а формули, позначення змінних та 
операцій (усі формули, усі позначення змінних та усі операції) беріть у 
знаки долара, ось так: $E = mc^2$, $a_1 = a^{(2)} \cdot a_{n, k}$, $e^x = 
\sum_{k = 0}^{\infty} {\frac{x^k}{k!}}$. Якщо вам 
не подобається, як \LaTeX подав формулу для експоненти (мені, наприклад, 
не подобається), то можна внести у код формули деякі корективи та написати ось так: $e^x 
= \sum\limits_{k = 0}^{\infty} {\dfrac{x^k}{k!}}$.

Для прикладу різні варіації коми у формулах: $(a, b)$ vs. $(a,b)$. Поки 
пакет \texttt{icomma} працює, різниця видна наочно.

Виключна формула (формула окремим рядком) робиться через спеціальне оточення з бекслешів та квадратових дужок або через оточення \texttt{equation}. Зауважте, що при цьому змінюється 
оформлення формул:
\[
e^x = \sum_{k = 0}^{\infty} {\frac{x^k}{k!}}\;.
\]
Оточення \texttt{equation} потрібне для створення нумерованих формул:
\begin{equation}
e^x = \sum_{k = 0}^{\infty} {\frac{x^k}{k!}}\;.
\label{eq:exponent}
\end{equation}

Потім на формули можна посилатися, наприклад, в \eqref{eq:exponent} наведено розвинення експоненти в ряд Маклорена.

Формули за замовчуванням не підтримують кирилічні літери. Зверніть увагу на 
порожній рядок перед попереднім реченням у tex-файлі: без нього не буде 
створено абзац.

Із більш специфічних позначень --- ось так, скажімо, можна подати 
перестановку:
$$\pi = \begin{pmatrix}
	1 & 2 & 3 & 4 & 5 & 6 & 7 & 8 & 9\\
	a & 5 & 9 & 6 & 4 & 8 & 2 & 1 & 7
\end{pmatrix},$$
де $a=3$. Зауважте, що у попередньому реченні нема порожнього рядочку 
перед <<де>> (та, відповідно, абзацу після формули), а кома внесена у 
виключну формулу, бо інакше вона переїде у наступний рядок тексту.

Декілька формул поспіль треба збирати в єдине ціле оточенням \texttt{align}; назви оточень із зірочками вказують \LaTeX'у не нумерувати формули. Наприклад, ось рекуренти для циклових чисел та чисел Стірлінга 
I~роду:
\begin{align*}
	c(n+1, k) &= c(n, k-1)+nc(n, k); \\
	s(n+1, k) &= s(n, k-1)-ns(n, k).
\end{align*}

Зверніть увагу на символ \verb|~| у попередньому абзаці tex-файлу; це нерозривний пробіл, який не дасть рознести пов'язані 
частини по різних рядках. Тільду треба ставити перед усіма посиланнями 
(команди ref та cite), перед тире та у місцях, які не можна розривати за 
правилами граматики.

Для специфічних позначень ви можете задавати власні команди (їх 
рекомендовано заносити у файл \texttt{02\_redefinitions}). Наприклад, 
подивіться, як оформлюється теорема Лагранжа-Бюрмана із використанням 
введених команд \texttt{Coef} та \texttt{compinv}:

\begin{theorem}[Лагранж, Бюрман] \label{thLagrangeBurmann}
	Для будь-якого ряду $A \in x \mathcal R[[x]]_1$ та $k \in \mathbb N$ справедливе співвідношення
	$$n \Coef[x^n] \left( \compinv{A}(x) \right)^k = k \Coef[x^{n-k}] \left(\! \frac{x}{A(x)} \!\right)^n.$$
\end{theorem}
\begin{proof}
	Доведення ви подивитесь деінде, а тут подивіться, як воно оформлюється 
	(зокрема, на квадратик наприкінці). 
\end{proof}

Іноді написаний файл треба компілювати двічі для одержання ефекту 
(скажімо, для коректної побудови усіх гіперпосилань та побудови змісту).  
Онлайн-сервіси на кшталт Overleaf справляються з такими ситуаціями за одну компіляцію. Однак той 
же Overleaf має звичку компілювати pdf-файли навіть за наявності помилок у 
тексті, просто ігноруючи відповідні місця. Якщо ви працюєте у Overleaf, 
то переконайтесь, що у вас нема червоних помилок після компіляції.

\section{(Назва третього підрозділу)}

Надамо деякі рекомендації щодо використання даного стильового файлу.

\begin{theorem}
	Використовуйте оточення \texttt{theorem} для теорем.
\end{theorem}
\begin{proof}
	Для доведень використовуйте оточення \texttt{proof}.
\end{proof}
\begin{theorem}
	Нумерація відбувається автоматично.
\end{theorem}
\begin{claim}
	Використовуйте оточення \texttt{claim} для тверджень.
\end{claim}
\begin{lemma}
	Використовуйте оточення \texttt{lemma} для лем.
\end{lemma}
\begin{corollary}
	Використовуйте оточення \texttt{corollary} для наслідків.
\end{corollary}
\begin{definition}
	Використовуйте оточення \texttt{definition} для визначень.
\end{definition}
\begin{example}
	Використовуйте оточення \texttt{example} для прикладів, на які є посилання.
\end{example}
\begin{remark}
	Використовуйте оточення \texttt{remark} для зауважень.
\end{remark}


\chapconclude

Наприкінці кожного розділу ви повинні навести коротенькі підсумки по його 
результатах. Зокрема, для оглядового розділу в якості висновків потрібно 
зазначити, які задачі у даній тематиці вже були розв'язані, а саме 
поставлена вами задача розв'язана не була (або розв'язана погано), тому у 
наступних розділах ви її й розв'язуєте.
%!TEX root = ../main.tex
% створюємо розділ
\chapter{(Назва третього розділу)}
\label{chap:theory}

\section{(Якийсь підрозділ)}

А от тут приклад таблички і може ще чогось))

Для подання матеріалів можна використовувати таблиці (наприклад, 
табл. \ref{tab_weight}). 

    \begin{table}[ht]
    \caption{Розрахунок якоїсь величини у декілька кроків}
    \label{tab_weight}
    \centering
        \begin{tabular}{|c|c|c|c|c|c|c|c|c|}
        \hline \multirow{2}{*}{Параметр $x_i$} & \multicolumn{4}{c|}{Параметр $x_j$} & 
            \multicolumn{2}{c|}{Перший крок} & \multicolumn{2}{c|}{Другий крок} \\
        \cline{2-9} & $X_1$ & $X_2$ & $X_3$ & $X_4$ & $w_i$ & 
            ${K_\text{в}}_i$ & $w_i$ & ${K_\text{в}}_i$ \\
        \hline $X_1$ & 1 & 1 & 1.5 & 1.5 & 5 & 0.31 & 19 & 0.32 \\
        \hline $X_2$ & 1 & 1 & 1.5 & 1.5 & 5 & 0.31 & 19 & 0.32 \\
        \hline $X_3$ & 0.5 & 0.5 & 1 & 0.5 & 2.5 & 0.16 & 9.25 & 0.16 \\
        \hline $X_4$ & 0.5 & 0.5 & 1.5 & 1 & 3.5 & 0.22 & 12.25 & 0.20 \\
        \hline \multicolumn{5}{|c|}{Разом:} & 16 & 1 & 59.5 & 1 \\
        \hline
        \end{tabular}
    \end{table}

Бажано, щоб кожен пункт завдань, окреслених у вступі, відповідав певному 
розділу або підрозділу у дипломній роботі.

\begin{theorem}
Нумерація у наступних розділах також проставляється автоматично та коректно.
\end{theorem}

\section{(Якийсь наступний підрозділ)}

Для подання матеріалів також дуже зручними є рисунки (наприклад, рис.
\ref{fig_sudak}).


\begin{figure}[ht]
\centering
    \begin{subfigure}[b]{0.5\textwidth}    
        \includegraphics[scale=0.3]{Images/Sudak.png}
        \caption{}
    \end{subfigure}%
    \begin{subfigure}[b]{0.5\textwidth}
        \includegraphics[scale=0.3]{Images/Tudak.png}
        \caption{}
    \end{subfigure}
 
    \caption{Різні види риб: (a) судак, (б) тудак}
    \label{fig_sudak}
\end{figure}

\chapconclude

Наприкінці розділу знову наводяться коротенькі підсумки.

%!TEX root = ../thesis.tex
\chapter{(Назва четвертого розділу)}
\label{chap:practice}

\section{(Якийсь підрозділ)}

Зазвичай четвертий розділ присвячено опису практичного застосування або 
експериментальної перевірки аналітичних результатів, одержаних у третьому 
розділі роботи. Втім, це не обов'язкова вимога, і структура основної 
частини диплому більш суттєво залежить від характеру поставлених завдань. 
Навіть якщо у вас є певне експериментальне дослідження, але його загальний 
опис займає дві сторінки, то краще приєднайте його підрозділом у 
попередній розділ.

При описі експериментальних досліджень варто:
\begin{itemize}
\item наводити повний опис експериментів, які проводились, параметрів 
обчислювальних середовищ, засобів програмування тощо;
\item наводити повний перелік одержаних результатів у чисельному вигляді для їх можливої 
перевірки іншими особами;
\item представляти одержані результати у вигляді таблиць та графіків, 
зрозумілих людському оку;
\item інтерпретувати одержані результати з погляду поставленої задачі 
та загальної проблематики ваших досліджень.
\end{itemize}

У жодному разі не потрібно вставляти у цей розділ тексти 
програм. Їх, як правило, наводять у Додатку А.


\chapconclude

Висновки до останнього розділу є, фактично, підсумковими під усім 
дослідженням; однак вони повинні стостуватись саме того, що розглядалось у 
розділі.

% Створюємо висновки
\conclusions
%!TEX root = ../main.tex
% створюємо Висновки до всієї роботи

А тут висновки до вашої роботи, можуть генеруватись з чатом гпт, як побажаєте,
але в ідеалі додайте:
\begin{enumerate}
    \item Якась фраза про те що мети було досягнуто
    \item Точні числа отримані під час роботи
    \item Підсумок навіщо воно було зроблене
\end{enumerate}


% Додаємо бібліографію
%%%%%%%% Якщо ви використовуєте bibtex, оце все треба розкоментувати звідси...
%\defbibenvironment{bibliography}
%  {\list
%     {\printfield[labelnumberwidth]{labelnumber}}
%     {\setfontsize{14}%
%     }}
%  {\endlist}
%  {\item}
%
%\printbibliography[heading=bibintoc,title={Перелік посилань}]
%%%%%%%% ...по осюди

% Замість бібтеху можна вписати та оформити бібліографію вручну
%!TEX root = ../thesis.tex
% створюємо список використаної літератури
\begin{thebibliography}    
    \bibitem{Chaffey}
    Chaffey D. Digital marketing: strategy, implemenation and practice / Dave Chaffey, Ellis-Chadwick Fiona. --- 6th ed. --- [S. l.] : Pearson, 2016. --- 702 p.
    
    \bibitem{Daniels}
    Daniels D. Email Marketing: An Hour a Day / David Daniels, Jeanniey Mullen. Indianapolis : Wiley Publishing, 2009. --- 291 p.
    
    \bibitem{Comrack}
    Comrack G. V. Email Spam Filtering A Systematic Review / Gordon V. Comrack. --- Hanover : Now Publishers, 2008. --- 136 p.
    
    \bibitem{manage}
    Manage domains --- Postmaster tools [Electronic resource] // Google Postmaster tools. --- Mode of access: \url{https://postmaster.google.com} (date of access: 29.04.2024). --- Title from screen.
    
    \bibitem{Wolford}
    Wolford B. How does the GDPR affect email? [Electronic resource] / Ben Wolford // GDPR.EU. --- Mode of access: \url{https://gdpr.eu/email-encryption/} (date of access: 28.04.2024). --- Title from screen.
    
    \bibitem{privacy}
    Privacy and Identity Management [Electronic resource] / ed. by M. Friedewald, S. Schiffner, S. Krenn. --- Cham : Springer International Publishing, 2021. --- Mode of access: \url{https://doi.org/10.1007/978-3-030-72465-8} (date of access: 28.04.2024). --- Title from screen.
    
    \bibitem{turn}
    Turn Emails into Revenue [Electronic resource] // Intuit Mailchimp. --- Mode of access: \url{https://mailchimp.com} (date of access: 11.05.2024). --- Title from screen.

 
\end{thebibliography}


% Створюємо додатки (дивись у файли додатків для пояснень)
% Якщо ви маєте меншу або більшу кількість додатків, модифікуйте ці рядки у відповідний спосіб
% Якщо ви не маєте додатків, просто закоментуйте ці рядки:
%!TEX root = ../main.tex
\append{Лістинги програм}
\label{appendix:A}

\section{Програма 1}

Лістинг файлу main.py

\lstinputlisting[language=Python]{Code/main.py}



Зауважте, як змінилась нумерація підрозділу.

%!TEX root = ../thesis.tex
\append{Презентація}
\label{appendix:B}

У додатку Б наводяться слайди презентації, з якою ви виступатимете на захисті.


% Нарешті
\end{document}