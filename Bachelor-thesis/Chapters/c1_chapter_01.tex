%!TEX root = ../thesis.tex

\chapter{Постановка задачі}
\label{chap:problem}  %% відмічайте кожен розділ певною міткою -- на неї наприкінці необхідно посилатись

Першим розділом будь-якої роботи бакалавра має бути розділ про постановку задачі.

Тут треба вказати, серед іншого, мету і завдання.

\textit{Метою дослідження} є певна абстрактна недосяжна річ на кшталт 
загальнолюдського щастя на горизонті. Для досягнення мети потрібно 
розв'язати \textit{задачу дослідження}, яка полягає у чомусь суттєво більш 
конкретному. Для розв'язання задачі необхідно вирішити такі завдання:
\begin{enumerate}
	\item провести огляд опублікованих джерел за тематикою дослідження;
	\item (наступний пункт, пов'язаний із теоретичним дослідженням);
	\item (і ще один, наприклад, про експериментальну перевірку результатів);
	\item (і взагалі, краще з керівником проконсультуйтесь, як ваші 
	завдання правильно писати).
\end{enumerate}

Також у цьому розділі варто викласти вимоги до розроблюваного програмного забезпечення (як функціональні, так і нефункціональні, наприклад час виконання чи точність прогнозування тощо).