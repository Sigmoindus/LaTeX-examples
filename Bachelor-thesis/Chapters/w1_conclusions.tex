%!TEX root = ../thesis.tex
% створюємо Висновки до всієї роботи
Загальні висновки до роботи повинні підсумовувати усі ваші досягнення у 
даному напрямку досліджень.

За кожним пунктом завдань, поставлених у вступі, у висновках повинен 
міститись звіт про виконання: виконано, не виконано, виконано частково (і 
чому саме так). Наприклад, якщо першим поставленим завданням у вас іде 
\quotes{огляд літератури за тематикою досліджень}, то на початку висновків ви 
повинні зазначити, що \quotes{у ході роботи був проведений аналіз 
опублікованих джерел за тематикою (...), який показав, що (...)}. Окрім 
простої констатації про виконання ви повинні навести, які саме результати 
ви одержали та проінтерпретувати їх з точки зору поставленої задачі, мети 
та загальної проблематики.

В ідеалі загальні висновки повинні збиратись з висновків до кожного 
розділу. Однак висновки не повинні містити 
формул, таблиць та рисунків. У висновках обов'язково повинні фігурувати конкретні числа
(на кшталт \quotes{нейронна мережа прогнозує з точністю $95.71\%$}).

Наприкінці висновків потрібно зазначити напрямки подальших досліджень: 
куди саме, як вам вважається, потрібно прямувати наступним дослідникам у 
даній тематиці.
