%!TEX root = ../thesis.tex
\chapter{(Назва четвертого розділу)}
\label{chap:practice}

\section{(Якийсь підрозділ)}

Зазвичай четвертий розділ присвячено опису практичного застосування або 
експериментальної перевірки аналітичних результатів, одержаних у третьому 
розділі роботи. Втім, це не обов'язкова вимога, і структура основної 
частини диплому більш суттєво залежить від характеру поставлених завдань. 
Навіть якщо у вас є певне експериментальне дослідження, але його загальний 
опис займає дві сторінки, то краще приєднайте його підрозділом у 
попередній розділ.

При описі експериментальних досліджень варто:
\begin{itemize}
\item наводити повний опис експериментів, які проводились, параметрів 
обчислювальних середовищ, засобів програмування тощо;
\item наводити повний перелік одержаних результатів у чисельному вигляді для їх можливої 
перевірки іншими особами;
\item представляти одержані результати у вигляді таблиць та графіків, 
зрозумілих людському оку;
\item інтерпретувати одержані результати з погляду поставленої задачі 
та загальної проблематики ваших досліджень.
\end{itemize}

У жодному разі не потрібно вставляти у цей розділ тексти 
програм. Їх, як правило, наводять у Додатку А.


\chapconclude

Висновки до останнього розділу є, фактично, підсумковими під усім 
дослідженням; однак вони повинні стостуватись саме того, що розглядалось у 
розділі.