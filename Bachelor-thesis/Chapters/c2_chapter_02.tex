%!TEX root = ../thesis.tex

\chapter{(Назва другого розділу)}
\label{chap:review}

\section{(Назва першого підрозділу)}

Другий розділ, як правило, повинен бути присвячений огляду попередніх результатів за 
тематикою вашого дослідження. У даному розділі повинні міститись всі 
визначення та описи, потрібні для дальшого викладення матеріалу, та результати 
ваших попередників.

Абсолютно на всі не ваші результати повинні стояти 
в належний спосіб оформлені посилання.

Розмір другого (оглядового) розділу не повинен перевищувати третини вашої 
дипломної роботи (без урахування додатків).


\section{(Назва другого підрозділу)}

Наведемо основні правила оформлення текстів у системі \LaTeX.

Для абзацу робіть пусті рядки у файлі. Курсивний текст робиться командою 
\texttt{textit}: \textit{ось так}. 

\quotes{Лапки} робляться командою \texttt{quotes}. Довге 
тире у тексті --- трьома дефісами, коротке -- двома дефісами; у формулах 
мінуси робляться одним дефісом: $a-b$.

Пишіть звичайний текст звичайним текстом, а формули, позначення змінних та 
операцій (усі формули, усі позначення змінних та усі операції) беріть у 
знаки долара, ось так: $E = mc^2$, $a_1 = a^{(2)} \cdot a_{n, k}$, $e^x = 
\sum_{k = 0}^{\infty} {\frac{x^k}{k!}}$. Якщо вам 
не подобається, як \LaTeX подав формулу для експоненти (мені, наприклад, 
не подобається), то можна внести у код формули деякі корективи та написати ось так: $e^x 
= \sum\limits_{k = 0}^{\infty} {\dfrac{x^k}{k!}}$.

Для прикладу різні варіації коми у формулах: $(a, b)$ vs. $(a,b)$. Поки 
пакет \texttt{icomma} працює, різниця видна наочно.

Виключна формула (формула окремим рядком) робиться через спеціальне оточення з бекслешів та квадратових дужок або через оточення \texttt{equation}. Зауважте, що при цьому змінюється 
оформлення формул:
\[
e^x = \sum_{k = 0}^{\infty} {\frac{x^k}{k!}}\;.
\]
Оточення \texttt{equation} потрібне для створення нумерованих формул:
\begin{equation}
e^x = \sum_{k = 0}^{\infty} {\frac{x^k}{k!}}\;.
\label{eq:exponent}
\end{equation}

Потім на формули можна посилатися, наприклад, в \eqref{eq:exponent} наведено розвинення експоненти в ряд Маклорена.

Формули за замовчуванням не підтримують кирилічні літери. Зверніть увагу на 
порожній рядок перед попереднім реченням у tex-файлі: без нього не буде 
створено абзац.

Із більш специфічних позначень --- ось так, скажімо, можна подати 
перестановку:
$$\pi = \begin{pmatrix}
	1 & 2 & 3 & 4 & 5 & 6 & 7 & 8 & 9\\
	a & 5 & 9 & 6 & 4 & 8 & 2 & 1 & 7
\end{pmatrix},$$
де $a=3$. Зауважте, що у попередньому реченні нема порожнього рядочку 
перед <<де>> (та, відповідно, абзацу після формули), а кома внесена у 
виключну формулу, бо інакше вона переїде у наступний рядок тексту.

Декілька формул поспіль треба збирати в єдине ціле оточенням \texttt{align}; назви оточень із зірочками вказують \LaTeX'у не нумерувати формули. Наприклад, ось рекуренти для циклових чисел та чисел Стірлінга 
I~роду:
\begin{align*}
	c(n+1, k) &= c(n, k-1)+nc(n, k); \\
	s(n+1, k) &= s(n, k-1)-ns(n, k).
\end{align*}

Зверніть увагу на символ \verb|~| у попередньому абзаці tex-файлу; це нерозривний пробіл, який не дасть рознести пов'язані 
частини по різних рядках. Тільду треба ставити перед усіма посиланнями 
(команди ref та cite), перед тире та у місцях, які не можна розривати за 
правилами граматики.

Для специфічних позначень ви можете задавати власні команди (їх 
рекомендовано заносити у файл \texttt{02\_redefinitions}). Наприклад, 
подивіться, як оформлюється теорема Лагранжа-Бюрмана із використанням 
введених команд \texttt{Coef} та \texttt{compinv}:

\begin{theorem}[Лагранж, Бюрман] \label{thLagrangeBurmann}
	Для будь-якого ряду $A \in x \mathcal R[[x]]_1$ та $k \in \mathbb N$ справедливе співвідношення
	$$n \Coef[x^n] \left( \compinv{A}(x) \right)^k = k \Coef[x^{n-k}] \left(\! \frac{x}{A(x)} \!\right)^n.$$
\end{theorem}
\begin{proof}
	Доведення ви подивитесь деінде, а тут подивіться, як воно оформлюється 
	(зокрема, на квадратик наприкінці). 
\end{proof}

Іноді написаний файл треба компілювати двічі для одержання ефекту 
(скажімо, для коректної побудови усіх гіперпосилань та побудови змісту).  
Онлайн-сервіси на кшталт Overleaf справляються з такими ситуаціями за одну компіляцію. Однак той 
же Overleaf має звичку компілювати pdf-файли навіть за наявності помилок у 
тексті, просто ігноруючи відповідні місця. Якщо ви працюєте у Overleaf, 
то переконайтесь, що у вас нема червоних помилок після компіляції.

\section{(Назва третього підрозділу)}

Надамо деякі рекомендації щодо використання даного стильового файлу.

\begin{theorem}
	Використовуйте оточення \texttt{theorem} для теорем.
\end{theorem}
\begin{proof}
	Для доведень використовуйте оточення \texttt{proof}.
\end{proof}
\begin{theorem}
	Нумерація відбувається автоматично.
\end{theorem}
\begin{claim}
	Використовуйте оточення \texttt{claim} для тверджень.
\end{claim}
\begin{lemma}
	Використовуйте оточення \texttt{lemma} для лем.
\end{lemma}
\begin{corollary}
	Використовуйте оточення \texttt{corollary} для наслідків.
\end{corollary}
\begin{definition}
	Використовуйте оточення \texttt{definition} для визначень.
\end{definition}
\begin{example}
	Використовуйте оточення \texttt{example} для прикладів, на які є посилання.
\end{example}
\begin{remark}
	Використовуйте оточення \texttt{remark} для зауважень.
\end{remark}


\chapconclude

Наприкінці кожного розділу ви повинні навести коротенькі підсумки по його 
результатах. Зокрема, для оглядового розділу в якості висновків потрібно 
зазначити, які задачі у даній тематиці вже були розв'язані, а саме 
поставлена вами задача розв'язана не була (або розв'язана погано), тому у 
наступних розділах ви її й розв'язуєте.