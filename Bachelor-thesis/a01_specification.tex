\pagestyle{empty}
\setlength{\parindent}{0cm}

{\centering
	\textbf{Національний технічний університет України}
	
	\textbf{\quotes{Київський політехнічний інститут імені Ігоря Сікорського}}

	\textbf{Факультет прикладної математики}
	
	\textbf{Кафедра прикладної математики}

}

{\raggedright
Рівень вищої освіти --- перший (бакалаврський)

Спеціальність --- 113~Прикладна математика

Освітньо-професійна програма \quotes{Наука про дані та математичне моделювання}

}

\addvspace{12pt} % невеликий відступ

\begin{flushright}
	\renewcommand{\arraystretch}{0.8}
	\begin{tabular}{l}
		\MakeUppercase{Затверджую} \\
		Завідувач кафедри\\
		\_\_\_\_\_\_\_\_\_\_\_\_\_~Олег ЧЕРТОВ \\
		\quotes{\_\_\_\_}~\_\_\_\_\_\_\_\_\_\_\_\_\_~\the\year~р. \\
	\end{tabular}
\end{flushright}

{\centering
	\textbf{\MakeUppercase{Завдання}}
	
	\textbf{на дипломну роботу студенту}
	
	%\addvspace{14pt}
	
	\reportAuthorGen

}

1. Тема роботи: \quotes{\reportTitle},
керівник роботи \supervisorFio, \supervisorRegalia,  затверджені наказом по університету \reportOrder.

2. Термін подання студентом роботи: \applicationDate

3. Вихідні дані до роботи: ???

4. Зміст роботи: ???

5. Перелік ілюстративного матеріалу: ???

%% Якщо немає консультантів, видаліть пункт 6. і зробіть пункт 7. пунктом 6.
6. Консультанти розділів роботи:

\addvspace{12pt}
\begin{center}
	\begin{tabularx}{\textwidth}{|C{0.2\textwidth}|L{0.4\textwidth}|C{0.15\textwidth}|C{0.15\textwidth}|}
		\hline
		Розділ & \multicolumn{1}{|C{0.4\textwidth}|}{Прізвище, ініціали та посада консультанта} & \multicolumn{2}{C{0.3\textwidth}|}{Підпис, дата} \\ \cline{3-4}
		&  & завдання видав & завдання прийняв \\
		\hline
		Математичне забезпечення & \consultFio, \consultRegalia & & \\
		\hline
	\end{tabularx}
\end{center}

\addvspace{12pt}

7. Дата видачі завдання: \assignmentDate

% Якщо перша частина завдання вилізе за сторінку --- приберіть команду \newpage
% Календарний план є продовженням завдання, а не окремою частиною
\newpage

\begin{center}
Календарний план

\addvspace{12pt}

    \begin{tabularx}{\textwidth}{|C{0.1\textwidth}|L{0.4\textwidth}|C{0.25\textwidth}|C{0.15\textwidth}|}
    \hline
    \No\ з/п & \multicolumn{1}{C{0.4\textwidth}|}{Назва етапів виконання дипломної роботи} & Термін виконання етапів роботи & Примітка \\
    \hline 
    % номер етапу
    1 & 
    % назва етапу
    Узгодження теми роботи із науковим керівником & 
    % термін виконання
    01-15 вересня \YearOfBeginning~р. &
    \\
%%% -- початок інтервалу для копіювання
    \hline 
    % номер етапу
    2 & 
    % назва етапу
    Огляд опублікованих джерел за тематикою дослідження & 
    % термін виконання
    Вересень-жовтень \YearOfBeginning~р. &
    \\
%%% -- кінець інтервалу для копіювання
% не прибирайте амперсанди та \\ наприкінці рядків!
% скопійовані інтервали вставляти перед фінальною \hline та заповнювати відповідно
% ось так:
%%% -- початок інтервалу для копіювання
    \hline 
    % номер етапу
    3 & 
    % назва етапу
    \ldots & 
    % термін виконання
    \ldots &
    \\
%%% -- кінець інтервалу для копіювання
    \hline %фінальна hline
    \end{tabularx}
\end{center}


\begin{tabular*}{\textwidth}{@{\extracolsep{\fill}}lrr}
	Студент &  \_\_\_\_\_\_\_\_\_\_\_\_\_ & \reportAuthorShort \\
	
	Керівник роботи & \_\_\_\_\_\_\_\_\_\_\_\_\_ & \supervisorFioShort \\
\end{tabular*}

\clearpage

\pagestyle{plain}

\setlength{\parindent}{1.25cm}
