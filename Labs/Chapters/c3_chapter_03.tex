%!TEX root = ../main.tex
% створюємо розділ
\chapter{(Назва третього розділу)}
\label{chap:theory}

\section{(Якийсь підрозділ)}

А от тут приклад таблички і може ще чогось))

Для подання матеріалів можна використовувати таблиці (наприклад, 
табл. \ref{tab_weight}). 

    \begin{table}[ht]
    \caption{Розрахунок якоїсь величини у декілька кроків}
    \label{tab_weight}
    \centering
        \begin{tabular}{|c|c|c|c|c|c|c|c|c|}
        \hline \multirow{2}{*}{Параметр $x_i$} & \multicolumn{4}{c|}{Параметр $x_j$} & 
            \multicolumn{2}{c|}{Перший крок} & \multicolumn{2}{c|}{Другий крок} \\
        \cline{2-9} & $X_1$ & $X_2$ & $X_3$ & $X_4$ & $w_i$ & 
            ${K_\text{в}}_i$ & $w_i$ & ${K_\text{в}}_i$ \\
        \hline $X_1$ & 1 & 1 & 1.5 & 1.5 & 5 & 0.31 & 19 & 0.32 \\
        \hline $X_2$ & 1 & 1 & 1.5 & 1.5 & 5 & 0.31 & 19 & 0.32 \\
        \hline $X_3$ & 0.5 & 0.5 & 1 & 0.5 & 2.5 & 0.16 & 9.25 & 0.16 \\
        \hline $X_4$ & 0.5 & 0.5 & 1.5 & 1 & 3.5 & 0.22 & 12.25 & 0.20 \\
        \hline \multicolumn{5}{|c|}{Разом:} & 16 & 1 & 59.5 & 1 \\
        \hline
        \end{tabular}
    \end{table}

Бажано, щоб кожен пункт завдань, окреслених у вступі, відповідав певному 
розділу або підрозділу у дипломній роботі.

\begin{theorem}
Нумерація у наступних розділах також проставляється автоматично та коректно.
\end{theorem}

\section{(Якийсь наступний підрозділ)}

Для подання матеріалів також дуже зручними є рисунки (наприклад, рис.
\ref{fig_sudak}).


\begin{figure}[ht]
\centering
    \begin{subfigure}[b]{0.5\textwidth}    
        \includegraphics[scale=0.3]{Images/Sudak.png}
        \caption{}
    \end{subfigure}%
    \begin{subfigure}[b]{0.5\textwidth}
        \includegraphics[scale=0.3]{Images/Tudak.png}
        \caption{}
    \end{subfigure}
 
    \caption{Різні види риб: (a) судак, (б) тудак}
    \label{fig_sudak}
\end{figure}

\chapconclude

Наприкінці розділу знову наводяться коротенькі підсумки.
